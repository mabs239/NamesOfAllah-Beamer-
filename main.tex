% compile with xelatex
 
\documentclass[17pt]{beamer}
\usetheme{Darmstadt}
\usepackage{polyglossia}
\usepackage{quran}

%\usepackage{fontspec} % for ttf font


\setdefaultlanguage[calendar=gregorian,numerals=maghrib]{arabic}
\setotherlanguage{english}
%\newfontfamily\arabicfont[Script=Arabic]{Microsoft Uighur}
%\newfontfamily\arabicfontsf[Script=Arabic]{Microsoft Uighur}

\newfontfamily\arabicfont[Script=Arabic, Scale=.9,Extension = .ttf]{msuighur}
\newfontfamily\arabicfontsf[Script=Arabic,Scale=.9, Extension = .ttf]{msuighur}

 
\setbeamertemplate{frametitle}[default][center]
 
\title[اسماءالحسنی]{أسماء الله الحسنى} % (optional, use only withNoto Nastaliq Urdu long paper titles)
 
\subtitle
{اللہ وتبارک تعالی کے پیارے  صفاتی نام فی القرآن}
 
\author[مہندس محمد ابو بکر صدیق]{مہندس محمد ابو بکر صدیق}% (optional, use only with lots of authors)
 
% - Give the names in the same order as the appear in the paper.
% - Use the \inst{?} command only if the authors have different
%   affiliation.
 
\institute[جامعہ ہندسیہ لاہور] % (optional, but mostly needed)
{
  رچنا کالج \\
  جامعہ ہندسیہ لاہور}
% - Use the \inst command only if there are several affiliations.
% - Keep it simple, no one is interested in your street address.
 
%\date
% - Either use conference name or its abbreviation.
% - Not really informative to the audience, more for people (including
%   yourself) who are reading the slides online
 
\subject{Theoretical Computer Science}
% This is only inserted into the PDF information catalog. Can be left
% out. 
 
% If you have a file called "university-logo-filename.xxx", where xxx
% is a graphic format that can be processed by latex or pdflatex,
% resp., then you can add a logo as follows:
 
% \pgfdeclareimage[height=0.5cm]{university-logo}{university-logo-filename}
% \logo{\pgfuseimage{university-logo}}
 
% for RTL liste
\makeatletter
\newcommand{\RTList}{\raggedleft\rightskip\@totalleftmargin}
\makeatother
 
 % RTL triangle for itemize
 \setbeamertemplate{itemize item}
 {\scriptsize\raise1.25pt\hbox{\donotcoloroutermaths$\blacktriangleleft$}} 
 
% Write Name, Ayat, Ayat Number and Surah Number and Surah Name in new slide
 
\ToggleAyahNumber
 
\newcommand{\nameSlide}[3]{
    \begin{frame}
        \frametitle{#1}
        \begin{enumerate}\RTList
            \item \quranayah[#2][#3] (#2-\surahname*[#2] :#3)
        \end{enumerate}
    \end{frame}
}
 
\newcommand{\nameSlidex}[5]{
    \begin{frame}
        \frametitle{#1}
        \begin{enumerate}\RTList
            \item \quranayah[#2][#3] (#2-\surahname*[#2] :#3)
            \item \quranayah[#4][#5] (#4-\surahname*[#4] :#5)
        \end{enumerate}
    \end{frame}
}
 
\newcommand{\nameSlidexx}[7]{
    \begin{frame}
        \frametitle{#1}
        \begin{enumerate}\RTList
            \item \quranayah[#2][#3] (#2-\surahname*[#2] :#3)
            \item \quranayah[#4][#5] (#4-\surahname*[#4] :#5)
            \item \quranayah[#6][#7] (#6-\surahname*[#6] :#7)
        \end{enumerate}
    \end{frame}
}
 
\newcommand{\nameSlidexxx}[9]{
    \begin{frame}
        \frametitle{#1}
        \begin{enumerate}\RTList
            \item \quranayah[#2][#3] (#2-\surahname*[#2] :#3)
            \item \quranayah[#4][#5] (#4-\surahname*[#4] :#5)
            \item \quranayah[#6][#7] (#6-\surahname*[#6] :#7)
            \item \quranayah[#8][#9] (#8-\surahname*[#8] :#9)
        \end{enumerate}
    \end{frame}
}
 
\begin{document}
\rightskip\rightmargin
 
\begin{frame}
  \titlepage
\end{frame}
 
\begin{frame}
\vspace{-2em}
عَنْ أَبِي هُرَيْرَةَ، قَالَ قَالَ رَسُولُ اللَّهِﷺ إِنَّ لِلَّهِ تَعَالَى تِسْعَةً وَتِسْعِينَ اسْمًا مِائَةً غَيْرَ وَاحِدَةٍ مَنْ أَحْصَاهَا دَخَلَ الْجَنَّةَ هُوَ اللَّهُ الَّذِي لاَ إِلَهَ إِلاَّ هُوَ الرَّحْمَنُ الرَّحِيمُ الْمَلِكُ الْقُدُّوسُ السَّلاَمُ الْمُؤْمِنُ الْمُهَيْمِنُ الْعَزِيزُ الْجَبَّارُ الْمُتَكَبِّرُ الْخَالِقُ الْبَارِئُ الْمُصَوِّرُ الْغَفَّارُ الْقَهَّارُ الْوَهَّابُ الرَّزَّاقُ الْفَتَّاحُ الْعَلِيمُ الْقَابِضُ الْبَاسِطُ الْخَافِضُ الرَّافِعُ الْمُعِزُّ الْمُذِلُّ السَّمِيعُ الْبَصِيرُ الْحَكَمُ الْعَدْلُ اللَّطِيفُ الْخَبِيرُ الْحَلِيمُ الْعَظِيمُ الْغَفُورُ الشَّكُورُ الْعَلِيُّ الْكَبِيرُ الْحَفِيظُ الْمُقِيتُ الْحَسِيبُ الْجَلِيلُ الْكَرِيمُ الرَّقِيبُ الْمُجِيبُ الْوَاسِعُ الْحَكِيمُ الْوَدُودُ الْمَجِيدُ الْبَاعِثُ الشَّهِيدُ الْحَقُّ الْوَكِيلُ الْقَوِيُّ الْمَتِينُ الْوَلِيُّ الْحَمِيدُ الْمُحْصِي الْمُبْدِئُ الْمُعِيدُ الْمُحْيِي الْمُمِيتُ الْحَىُّ الْقَيُّومُ الْوَاجِدُ الْمَاجِدُ الْوَاحِدُ الصَّمَدُ الْقَادِرُ الْمُقْتَدِرُ الْمُقَدِّمُ الْمُؤَخِّرُ الأَوَّلُ الآخِرُ الظَّاهِرُ الْبَاطِنُ الْوَالِي الْمُتَعَالِي الْبَرُّ التَّوَّابُ الْمُنْتَقِمُ الْعَفُوُّ الرَّءُوفُ مَالِكُ الْمُلْكِ ذُو الْجَلاَلِ وَالإِكْرَامِ الْمُقْسِطُ الْجَامِعُ الْغَنِيُّ الْمُغْنِي الْمَانِعُ الضَّارُّ النَّافِعُ النُّورُ الْهَادِي الْبَدِيعُ الْبَاقِي الْوَارِثُ الرَّشِيدُ الصَّبُورُ ‏
(جامع ترمذی :بَاب مَا جَاءَ فِي فَضْلِ الدُّعَاءِ)
\end{frame}
 
\nameSlidexx{ٱلْرَّحْمَـٰنُ}{2}{163}{13}{30}{17}{110}
\nameSlidex{ٱلْرَّحِيْمُ}{4}{64}{2}{143}
 
\nameSlidexx{ٱلْمَـالِكُ } {59}{23} {20}{114} {23}{116}
 
\nameSlidex{ٱلْقُدُّوسُ }{59}{23}{62}{1}
\nameSlide{ٱلْسَّلَامُ}{59}{23}
\nameSlide{ٱلْمُؤْمِنُ}{59}{23}
\nameSlide{ٱلْمُهَيْمِنُ}{59}{23}
\nameSlidexxx{ٱلْعَزِيزُ}{3}{6}{4}{158}{9}{40}{48}{7}
\nameSlide{ٱلْجَبَّارُ }{59}{23}
\nameSlide{ٱلْمُتَكَبِّرُ}{59}{23}
\nameSlidex{ٱلْخَالِقُ }{6}{102}{13}{16}
 
\nameSlide{ٱلْبَارِئُ }{59}{24}
\nameSlide{ٱلْمُصَوِّرُ }{59}{24}
\nameSlidexx{ٱلْغَفَّارُ }{20}{82}{38}{66}{39}{5}
\nameSlidexx{ٱلْقَهَّارُ }{12}{39}{13}{16}{14}{48}
\nameSlidex{ٱلْوَهَّابُ }{38}{9}{38}{35}
\nameSlide{ٱلْرَّزَّاقُ }{51}{58}
\nameSlide{ٱلْفَتَّاحُ }{34}{26}
\nameSlidexx{ٱلْعَلِيمُ }{2}{158}{3}{92}{4}{35}
\nameSlide{ٱلْقَابِضُ }{2}{245}
\nameSlide{ٱلْبَاسِطُ }{2}{245}
\nameSlide{ٱلْخَافِضُ }{56}{3}
\nameSlidex{ٱلْرَّافِعُ }{58}{11}{6}{83}
\nameSlide{ٱلْمُعِزُّ }{3}{26}
\nameSlide{ٱلْمُذِلُّ }{3}{26}
\nameSlidexx{ٱلْسَّمِيعُ }{2}{127}{2}{256}{8}{17}
\nameSlidexx{ٱلْبَصِيرُ }{4}{58}{17}{1}{42}{11}
\nameSlide{ٱلْحَكَمُ }{22}{69}
\nameSlide{ٱلْعَدْلُ }{6}{115}
\nameSlidexx{ٱلْلَّطِيفُ }{22}{63}{31}{16}{33}{34}
\nameSlidexx{ٱلْخَبِيرُ }{6}{18}{17}{30}{49}{13}
\nameSlidexx{ٱلْحَلِيمُ }{2}{225}{17}{44}{22}{59}
 
\nameSlidexx{ٱلْعَظِيمُ }{2}{255}{42}{4}{56}{96}
\nameSlidexx{ٱلْغَفُورُ }{2}{173}{8}{69}{16}{110}
\nameSlidexx{ٱلْشَّكُورُ }{35}{30}{35}{34}{42}{23}
\nameSlidexx{ٱلْعَلِيُّ }{4}{34}{31}{30}{42}{4}
\nameSlidexx{ٱلْكَبِيرُ }{13}{9}{22}{62}{31}{30}
\nameSlidexx{ٱلْحَفِيظُ }{11}{57}{34}{21}{42}{6}
\nameSlide{ٱلْمُقِيتُ }{4}{85}
\nameSlidexx{ٱلْحَسِيبُ }{4}{6}{4}{86}{33}{39}
\nameSlidex{ٱلْجَلِيلُ }{55}{27}{7}{143}
\nameSlidex{ٱلْكَرِيمُ }{27}{40}{82}{6}
\nameSlidex{ٱلْرَّقِيبُ }{4}{1}{5}{117}
\nameSlide{ٱلْمُجِيبُ }{11}{61}
\nameSlidexx{ٱلْوَاسِعُ }{2}{268}{3}{73}{4}{130}
 
\nameSlidexx{ٱلْحَكِيمُ }{31}{27}{46}{2}{57}{1}
\nameSlidex{ٱلْوَدُودُ }{11}{90}{85}{14}
\nameSlide{ٱلْمَجِيدُ }{11}{73}
\nameSlidex{ٱلْبَاعِثُ }{22}{7}{2}{246}
 
\nameSlidexx{ٱلْشَّهِيدُ }{4}{166}{22}{17}{41}{53}
\nameSlidexx{ٱلْحَقُّ }{6}{62}{22}{6}{23}{116}
%\nameSlidexx{ٱلْوَكِيلُ }{4}{130}{4}{171}{28}{28}
\nameSlidexx{ٱلْوَكِيلُ }{3}{173}{4}{132}{28}{28}
 
\nameSlidexx{ٱلْقَوِيُّ }{22}{40}{22}{74}{42}{19}
\nameSlide{ٱلْمَتِينُ }{51}{58}
\nameSlidexx{ٱلْوَلِيُّ }{4}{45}{7}{196}{42}{28}
\nameSlidexx{ٱلْحَمِيدُ }{14}{8}{31}{12}{31}{26}
\nameSlidex{ٱلْمُحْصِيُ }{72}{28}{78}{29}
\nameSlidexx{ٱلْمُبْدِئُ }{10}{34}{27}{64}{29}{19}
\nameSlidexx{ٱلْمُعِيدُ }{10}{34}{27}{64}{29}{19}
\nameSlidexx{ٱلْمُحْيِي }{7}{158}{15}{23}{30}{50}
\nameSlidexx{ٱلْمُمِيتُ }{3}{156}{7}{158}{15}{23}
\nameSlidexx{ٱلْحَىُّ }{2}{255}{3}{2}{20}{111}
\nameSlidexx{ٱلْقَيُّومُ }{2}{255}{3}{2}{20}{111}
\nameSlidexx{ٱلْوَاجِدُ }{38}{44}{7}{102}{24}{39}
 
\nameSlidex{ٱلْمَاجِدُ }{85}{15}{11}{73}
\nameSlidexx{ٱلْوَاحِدُ }{13}{16}{14}{48}{38}{65}
\nameSlide{ٱلْأَحَد }{112}{1}
\nameSlide{ٱلْصَّمَدُ }{112}{2}
\nameSlidexx{ٱلْقَادِرُ }{6}{65}{46}{33}{75}{40}
\nameSlidexx{ٱلْمُقْتَدِرُ }{18}{45}{54}{42}{6}{65}
\nameSlidex{ٱلْمُقَدِّمُ }{16}{61}{10}{2}
 
\nameSlide{ٱلْمُؤَخِّرُ }{71}{4}
\nameSlide{ٱلأَوَّلُ }{57}{3}
\nameSlide{ٱلْآخِرُ }{57}{3}
\nameSlide{ٱلْظَّاهِرُ }{57}{3}
\nameSlide{ٱلْبَاطِنُ }{57}{3}
\nameSlide{ٱلْوَالِي }{13}{11}
\nameSlide{ٱلْمُتَعَالِي }{13}{9}
\nameSlide{ٱلْبَرُّ }{52}{28}
\nameSlidexx{ٱلْتَّوَّابُ }{2}{128}{4}{64}{49}{12}
\nameSlidexx{ٱلْمُنْتَقِمُ }{32}{22}{43}{41}{44}{16}
\nameSlidexx{ٱلْعَفُوُّ }{4}{43}{4}{99}{4}{149}
\nameSlidexx{ٱلْرَّؤُفُ }{9}{117}{57}{9}{59}{10}
\nameSlide{مَالِكُ ٱلْمُلْكُ }{3}{26}
\nameSlidex{ذُو ٱلْجَلَالِ وَٱلْإِكْرَامُ}{55}{27}{55}{78}
\nameSlide{ٱلْمُقْسِطُ }{3}{18}
\nameSlide{ٱلْجَامِعُ }{3}{9}
\nameSlidexx{ٱلْغَنيُّ }{39}{7}{47}{38}{57}{24}
\nameSlide{ٱلْمُغْنِيُّ }{9}{28}
%\nameSlidexx{ٱلْمَانِعُ }{}{}{}{}{}{}
\nameSlidex{ٱلْمَانِعُ }{9}{54}{17}{59}
\nameSlide{ٱلْضَّارُ }{6}{17}
%\nameSlide{ٱلْنَّافِعُ }{30}{37}
\nameSlide{ٱلْنَّافِعُ }{22}{28}
 
\nameSlide{ٱلْنُّورُ }{24}{35}
 
\nameSlide{ٱلْهَادِي }{22}{54}
\nameSlidex{ٱلْبَدِيعُ }{2}{117}{6}{101}
\nameSlide{ٱلْبَاقِي }{55}{27}
\nameSlidex{ٱلْوَارِثُ }{15}{23}{57}{10}
 
%\nameSlidexx{ٱلْرَّشِيدُ }{}{}{}{}{}{}
\nameSlidex{ٱلْرَّشِيدُ }{11}{87}{72}{10}
 
\nameSlidexx{ٱلْصَّبُورُ }{2}{153}{3}{200}{103}{3}
 
\end{document}